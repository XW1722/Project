\documentclass[11pt, oneside]{article}   	% use "amsart" instead of "article" for AMSLaTeX format

\usepackage{geometry}                		% See geometry.pdf to learn the layout options. There are lots.
\geometry{letterpaper}                   		% ... or a4paper or a5paper or ... 
%\geometry{landscape}                		% Activate for rotated page geometry
%\usepackage[parfill]{parskip}    		% Activate to begin paragraphs with an empty line rather than an indent
\usepackage{graphicx}				% Use pdf, png, jpg, or eps§ with pdflatex; use eps in DVI mode								% TeX will automatically convert eps --> pdf in pdflatex		
\usepackage{amssymb}
\usepackage{pgfgantt}
\usepackage{float}
\usepackage{natbib}
%SetFonts

%SetFonts

\title{Placing UK Research within the International STEM Funding Landscape}

\begin{document}


\begin{titlepage}
\newcommand{\HRule}{\rule{\linewidth}{0.5mm}}
\center
\textsc{\LARGE MSc Project Proposal}\\[1.5cm] % Name of your university/college
\textsc{\Large Imperial College London}\\[0.5cm] % Major heading such as course name
\textsc{\large Computational Methods in Ecology and Evolution}\\[0.5cm] % Minor heading such as course title
%----------------------------------------------------------------------------------------
%	TITLE SECTION
%----------------------------------------------------------------------------------------
\makeatletter
\HRule \\[0.4cm]
{ \huge \bfseries \@title}\\[0.4cm] % Title of your document
\HRule \\[1.5cm]
 
%----------------------------------------------------------------------------------------
%	AUTHOR SECTION
%----------------------------------------------------------------------------------------

\begin{minipage}{0.4\textwidth}
\begin{flushleft} \large
\emph{Author:}\\
Xuan Wang % Your name
\end{flushleft}
\end{minipage}
~
\begin{minipage}{0.4\textwidth}
\begin{flushright} \large
\emph{Leading Supervisor:} \\
Dr. Samraat S. Pawar \\[1.2em] % Supervisor's Name
\emph{Other Supervisor:} \\
W. D. Pearse
\end{flushright}
\end{minipage}\\[2cm]
\makeatother

{\large \today}\\[2cm] % Date, change the \today to a set date if you want to be precise

\vfill % Fill the rest of the page with whitespace

\end{titlepage}
					% Activate to display a given date or no date

\maketitle

\section{Introduction and Proposed Questions}

The funding strategy for STEM discipline differs among countries, leading to differences in their development and contribution to the world. Nevertheless, it is not uncommon that bias sometimes affects the distribution of research funding [\cite{crudden2022gender}]. Therefore, studying the base of bias in research funding could be beneficial to understand the current research funding distribution and gaining a more objective result.

\bigbreak
\noindent In my project, we aim to find the answer to the question: \textit{Is there any bias within the international STEM funding landscape?} In particular, this question will be answered from the following aspects:

\begin{itemize}
\item \textit{Are the research councils funding all the fields equally?}
\bigbreak
Research councils usually claim that they have funded the projects equally, and therefore the first thing to study is the fairness of the research funding. Our project will study whether the funding is spread equally across all the available research fields in the UK.
\item \textit{If not, which are the most and the least funded research fields?}
\bigbreak If the result of the last question reveals a negative result indicating an unequal distribution across the research fields, it will then be considered which research area is getting more funding and which is getting less.
\item \textit{What are the potential causes of the biases?}
\bigbreak There are several common risks in research funding, such as gender bias, risk aversion bias, racial bias, etc [\cite{wojick2015government}]. Due to financial constraints, the ``risk" of a proposal is usually a critical point in whether the project will be funded [\cite{franzoni2022funding}]; The funding gap between different genders and races or ethnicities have also been proved by the research of Romy Lee \& Naomi Ellemiers [\cite{van2015gender}] and that of Konkel [\cite{konkel2015racial}]. Our report will examine which type of bias is causing inequality in the research funding landscape.
\end{itemize}


\section{Methodology}
Fine-scale data will be used for our project. Mallet will be applied for the Machine Learning procedure. Our report will use data from the UK Research and Innovation (UKRI).
\bigbreak
\noindent We will apply pre-processing to the raw data, and the clean fine-scale data will be used for the analysis. HPC could be employed if the data is too big; the results from different countries will be compared for the final analysis.

\section{Anticipated outputs and outcomes}

\begin{itemize}
\item The bias in research funding in a set of countries will be displayed concerning countries respectively;
\item For each type of bias, the result for different countries will be taken into comparison, and the difference in the results of developed and developing countries will also be studied;
\item Analysis will be conducted regarding the potential reason and consequence of the biases.

\end{itemize}

\section{Project Feasibility}

The timeline for this project is displayed as follows:

\begin{figure}[htbp]

\begin{center}

\begin{ganttchart}[y unit title=0.4cm,
y unit chart=0.5cm,
vgrid,hgrid, 
title label anchor/.style={below=-1.6ex},
title left shift=.05,
title right shift=-.05,
title height=1,
progress label text={},
bar height=0.7,
group right shift=0,
group top shift=.6,
group height=.3]{1}{20}
%labels
\gantttitle{Month}{20} \\
\gantttitle{April}{4} 
\gantttitle{May}{4} 
\gantttitle{June}{4} 
\gantttitle{July}{4} 
\gantttitle{August}{4} \\
%tasks
\ganttbar{Review literature} {1}{4}\\
\ganttbar{Find data}{3}{5} \\
\ganttbar{Data pre-processing}{5}{8} \\
\ganttbar{Applying machine learning approaches}{9}{16} \\
\ganttbar{Write up}{5}{20} \\



\end{ganttchart}
\end{center}
\caption{Gantt Chart}
\end{figure}

\noindent The writing will be covered throughout most of the time. The aim is to complete the first write-up by early August, and then modify it for the rest of the weeks. 

\pagebreak

\bibliographystyle{agsm}
\bibliography{Proposal}

\end{document}  