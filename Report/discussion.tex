\section{Discussion}

The findings of this study point to a gender disparity in the overall pattern of research funding within the UK STEM sector, no matter the aspect of funding amount or funding ratio. A positive aspect is the partial reduction of this gap, which could be attributed to the growing focus on funding diversity. However, the beneficial impact is not evident and inequities persist. On the UKRI's official website, a commitment to gender equality is outlined, detailing the equality plan from 2022 to 2026. This plan encompasses transparency and monitoring of gender data, support for female leaders in the UK, efforts to address gender inequality in recruitment, etc. [\cite{plan}] However, to my knowledge, no actions in the plan specifically and directly address fair research funding between genders. Access to these resources remains unequal among different genders.\\
\\
This study also compared the trend across STEM categories over the years. The results showed several observations: 1) Despite the general trend of reducing the gender gap, the gap in Chemistry increased in recent years; 2) The gender gap in the funding ratio in Biosciences persists; 3) Though most engineering disciplines have a gender gap in funding amount, the gap in funding ratio is small; 4) The gap in Veterinary Science shows an opposite pattern, where females receive higher average amount of funding but lower funding ratio compared to males.

The findings indicate a decline in the gap across most categories, with the exception of Chemistry, where the disparities in both funding amount and funding ratio in recent years have grown compared to 2015. According to a 2018 report by the Royal Society of Chemistry, the relative percentage of female chemists advancing from undergraduate studies to senior academic roles has decreased by 35 percentage points [\cite{RoyalChemistry}]. The primary causes identified include arbitrary funding criteria and a lack of transparency in the recruitment and promotion process [\cite{RoyalChemistry}]. Despite efforts to reduce the gender gap [\cite{RSCFramework2023}], the gap only slightly decreased and continued beyond 2018. Biosciences exhibit a similar trend; the persistent funding ratio gap reveals no notable increase in the gender gap. Positively, there is no apparent gender disparity in the funding amount for funded researchers. It was noted that female researchers in biosciences are generally less successful in obtaining research funding. Previous research has shown that women with biological sciences degrees are less likely than men to work as scientists post-degree and tend to have shorter publishing careers, primarily due to leaving the field, and lower publishing rates as well [\cite{malespina2023gender}]. Contributing factors may include: 1. Gender disparities in the recruitment and promotion of women in biological sciences, similar to chemistry; 2. Differences in life goals between males and females may influence their decision-making [\cite{elliott2016editor}].\\

Another key result from my analyses is the gender disparity in funding amounts for engineering disciplines. Unlike the situation in Biosciences, the gender gap here is mainly observable in the amounts awarded, while gender equality in the funding ratio is relatively higher. This suggests that the funding structure could be a key factor in the gender imbalance in this sector. Based on information from the UKRI Engineering and Physical Sciences Research Council (EPSRC) \cite{UKRIengineering}, there are significant differences in the sizes of grants applied for by different genders, with female researchers consistently applying for smaller grants. Furthermore, the report points out that from 2007 to 2018, the award rates by value became increasingly different as the grant amounts increased \cite{UKRIengineering}. The award rate for women remained consistently around 30\% for all grant amounts, whereas for men it rose progressively from about 30\% to nearly 60\% as the grant amount increased. An EPSRC survey of female applicants revealed several reasons that might act as barriers to applying for larger grants \cite{UKRIsurvey}: 1. Lack of time due to being overwhelmed by other responsibilities; 2. Unfair biases in the peer review process; 3. Entry requirements for applications, such as institutional ranking or existing grant portfolio; 4. Absence of institutional support for large grant proposals. Notably, two of these primary causes are institution-related, which could indicate a troubling situation for female researchers. Previous research has shown that women are generally less likely to secure positions in highly ranked institutions compared to men \cite{gender_science}. Consequently, even though higher-ranked institutions may offer benefits to their female staff, the barriers to entering these institutions might prevent many women from accessing these advantages. This highlights potential inequities in resource allocation and possible biases in admission criteria.\\

The funding landscape in Veterinary Science reveals contrasting patterns: The average amount of funding for female researchers is generally higher than for male researchers, particularly in the first three years; however, male researchers have a higher funding ratio. This may be related to the significant representation of women in veterinary science, as evidenced by data from the Royal College of Veterinary Surgeons, which reported that in 2021, 77\% of practising veterinary surgeons in the UK were women [\cite{RCVS}]. Nevertheless, women are significantly underrepresented in journal publications - In 2023, 68\% of authors in a main journal of veterinary science, namely veterinary surgery, were men\cite{edinburgh_veterinary}]. The gendered masculine culture and the barriers mentioned in the previous paragraphs could be potential reasons for this phenomenon [\cite{liu2021women}]. In addition to the comparatively higher interest in this field by women, veterinary science is often considered a caring profession. Previous literature has proved the gendered nature in care professions, with women being more prevalent in part due to gender stereotypes such as men being less compassionate but more professional than women [\cite{poole1997caring}]. This also sheds light on why fewer women pursue academic careers in Veterinary Science, despite their high participation and graduation rates in veterinary programs, and why female academics are more likely to hold lower-ranked positions compared to their male colleagues [\cite{liu2021women}].
\\
\subsection{Implications and Recommendations}

In conclusion, our results revealed a significant gender bias in STEM research funding within the UK. Although there is a general decreasing trend for the gap, the academic framework continues to pose challenges to women. The internal link between research funding and the ranking of their institution could pose additional hurdles for female researchers. This could be regarded as a systematic bias in the review criteria, which benefits male applicants due to previous accumulative advantage [\cite{Holly2019}]. This inequality not only causes a reduction in the female researcher's incentive to deliver innovative studies that contribute to science, but would also be a factor influencing the career life of female researchers [\cite{jebsen2022dismantling}]. The number of research fundings, as an important role that affects the promotion of academic careers, would directly impact the retention and progression of female researchers [\cite{jebsen2022dismantling}], forming a vicious circle harmful to the situation of female researchers. To break the cycle, a systematic improvement in gender equality is necessary, which could include establishing a diversity focus group that monitors the diversity of the employment of each institution and ensures that the university recruitment process is fair and offers the same opportunity to various groups of people [\cite{sardelis2017ten}]. At the same time, this focus group should also be responsible for revising the funding application criteria, ensuring that the criteria could minimize the impact of accumulated advantage. 

\subsection{Limitations}
In this study, we conducted the trend comparison using the ratio of each gender getting funded, where the HESA staff data is used for the calculation. We assumed that the HESA staff number in each classification and university is the total number of researchers in these areas. In addition, another caveat that is worth noticing is that this study only considered females and males as the genders. All the other genders are not listed in consideration, while these are also worth more attention. Though this study mainly focuses on female researchers, it can be a broad indication of the situation of most minority groups [Jebsen et al. (2022)]. Lastly, the gender determination for each project is based on the name of the primary applicant, which could introduce some inaccuracy.\\
\\
\subsection{Future directions}

In this study, I have shown that there are gender biases among STEM disciplines, considering the total number of HESA staff to account for bias at all stages. However, to check the existing problem in the current diversity system, it would also be beneficial to study the funding trends based on the total number of applications so that we can find the proportion of bias caused during different stages, including the bias in the review stage and the step before application. In addition, this study only learned about the gender gap in STEM research funding. According to previous research, there may also be many other biases that may exist in research funding, including racial or nationality bias, ethnicity bias, risk aversion bias, etc. [\cite{wojick2015government}] All diversity should be as important as each other [\cite{LSE2018}], and bias in any aspect would have a direct impact on science innovation and career life of individuals. As such, more studies on other elements of discrimination should be conducted to increase awareness. 
