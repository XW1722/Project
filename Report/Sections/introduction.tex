\section{Introduction}
\subsection{Background}

\noindent STEM - for Science, Technology, Engineering, Mathematics - has become necessary due to its power in solving global challenges such as climate change, resource availability, and human health [\cite{Flavia}]. As a traditional global problem to be discussed [\cite{garcia2019bridging}], an increasing trend of research has also been done, studying the diversity gap in various aspects of the STEM area (employment, salary, etc.) to ensure the payoffs from STEM is beneficial worldwide. Nevertheless, it is still questionable whether these studies themselves are fair - according to the evidence provided by \cite{handley2015quality}, studies by males are usually less likely to judge gender discrimination than females. The results within a biased academic environment can be irrelevant, or even harmful, to women [\cite{cislak2018bias}]. The equality for each gender in funding application would therefore be an important factor influencing the objectiveness of the research, which has a non-negligible impact on the incentive of qualified applicants to engage in the studies. Another important fact is that the disadvantage of females in the academic system could also lead to an accumulated bias, known as the "Mathew Effect", according to previous literature [\cite{jebsen2020review}]. The employment criteria for application can be an example since evidence shows that women are less favourable than men for granting a position at higher-ranking institutions, which would disadvantage females when submitting an application [\cite{cruz2022gender}]. This inequality, therefore, causes an underrepresentation of females in the academic field, which further brings a more intense situation of gender discrimination.\\
\\
As one of the top global corporate Research \& Development investors [\cite{GII2022}], a £39.8 billion R\&D budget has been announced by the UK government for 2022-2025 to stimulate the power in science development [\cite{funding2025}]. On the other hand, the Engineering and Physical Sciences Research Council (EPSRC) data in 2016-2017 showed that £944m funding was awarded to male applicants while only £69m went to female applicants [\cite{jebsen2020review}]. An unequal opportunity for research funding would reduce the incentive and access to institutional resources [\cite{cruz2022gender}]. Therefore, my study aims to determine whether there is a gender bias in the research funding in the UK, specifically in the STEM area.

\subsection{Research aims and approaches}

My study will use the data collected from the UK Research and Innovation (UKRI) from 2015 to 2022 to do the analysis. The study used a machine learning method, topic analysis, to the whole dataset and the results were classified using ChatGPT. The classified results were then analysed and compared among different subjects and universities.

\subsubsection{Research questions}

In this report, I will examine the overall pattern of gender proportion and then make a comparison across various categories and institutions. My primary focus will be to answer the following questions:
\begin{itemize}
    \item From 2015 to 2022, how has the funding trend for each gender evolved? Is there a noticeable disparity?
    \item If a gender disparity exists, what could be the underlying cause?
    \item Is there any implication of the bias?
\end{itemize}
\bigbreak
\noindent To address the first question, this report will assess the temporal trends of funding amount over time for each gender, and a two-sample t-test will be employed for evidence. The subsequent questions will be explored by comparing the trend among different classifications and universities, considering recent policies and previous literature.
