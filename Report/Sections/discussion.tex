\section{Discussion}

This study's results have indicated a gender gap in the general trend of research funding in the STEM field in the UK. A positive sign is that the gap is being reduced to some extent, possibly due to the increasing attention paid to the diversity of research funding. Nevertheless, the positive impact is not apparent, and inequality is still observed. On the official website of UKRI, a commitment to gender equality is published, establishing the equality plan from 2022 to 2026. This includes ensuring the transparency and monitoring of the gender data, providing support to UK female leaders, targeting to address gender inequality in the recruitment process, etc. [\cite{plan}] However, to the best of our knowledge, there isn't any plan specifically for the fairness of research funding for different genders currently, which may be the potential reason leading to our result that the bias exists in the funding amount for males and females. This reveals that despite the government's announcement of increasing the funding for STEM subjects, different genders do not have equal access to the resources. \\
\\
Having done the data analysis of the funding amount of each gender, I compared the trend among the classifications in the STEM field and among different universities. In a few classifications, mainly the engineering disciplines, the difference between genders is comparatively smaller; a trend of reducing the gap is observed in most other classifications, except Chemistry, where the difference is more significant than the previous value in 2015. A report by the Royal Society of Chemistry in 2018 indicates that the relative proportion of female chemists between undergraduate study and reaching senior positions in academia has dropped by 35 percentage points [\cite{RoyalChemistry}]. The funding structure would be one of the leading causes of this phenomenon, where arbitrary funding criteria are mentioned; another cause is the lack of transparency in the recruitment and promotion process [\cite{RoyalChemistry}]. Though some actions were taken to mitigate the gender gap [\cite{RSCFramework2023}], the gap was minorly reduced initially and persisted after 2018.\\
\\
While the comparison among universities confirms the presence of gender bias across all the studied institutions, a minor difference is still observed. In the universities with higher ranks, the gender gap tends to be comparatively more minor in general. This could point to a concerning situation for female researchers - According to previous studies, females are usually less likely to be granted a position in institutions with higher ranking compared to males [\cite{gender_science}]. Consequently, even if higher-ranked institutions offer advantages to their female staff, the barriers to entry for these institutions might prevent many women from enjoying these benefits. This underscores potential inequities in resource distribution and possible biases in admission criteria.\\
\\
\subsection{Implications and Recommendations}

In conclusion, our results revealed a significant gender bias in STEM research funding within the UK.While there is a decreasing trend of the gap, the academic framework continues to pose challenges for females. The internal link between the research funding and the ranking of their institution could pose additional hurdles for female researchers. This could be regarded as a systematic bias in the reviewing criteria, which benefits male applicants due to previous accumulative advantage [\cite{Holly2019}]. The inequality not only causes a reduction in the female researcher's incentive to deliver innovative studies contributing to science, but would also be a factor influencing the career life of female researchers [\cite{jebsen2022dismantling}]. The number of research fundings, as an important role affecting the promotion of academic careers, would directly impact the retention and progression of female researchers [\cite{jebsen2022dismantling}], forming a vicious circle harmful to female researchers' situation. To break the cycle, having a systematic improvement for gender equality is necessary, which could include establishing a diversity focus group that keeps an eye on the diversity of each institution's employment and ensures that the recruitment process for universities is fair and offers equal opportunity to various groups of people [\cite{sardelis2017ten}]. At the same time, this focus group should also be responsible for revising the funding application criteria, ensuring that the criteria could minimise the impact of accumulated advantage. \\
\subsection{Limitations}
In this study, we conducted the trend comparison using the ratio of each gender getting funded, where the HESA staff data is used for the calculation. We assumed that the HESA staff number in each classification and university is the total number of researchers in these areas. In addition, another caveat that is worth noticing is that this study only considered females and males as the genders. All the other genders are not listed in consideration, while these are also worth more attention. Though this study mainly focuses on female researchers, it can be a broad indication of the situation of most minority groups [Jebsen et al. (2022)]. Lastly, the gender determination for each project relies on the name of the primary applicant, which could introduce some inaccuracy.\\
\subsection{Future Work}

In this research, I have conducted the university-wide comparison and also showed the difference among disciplines based on the total number of HESA staff to consider the bias in all stages. Nevertheless, to check the existing problem in the current diversity system, it would also be beneficial to study the ratio based on the total number of applications so that we can find the proportion of bias caused during different stages, including the bias in the review stage and the step before application. In addition, this study only learned about the gender gap in STEM research funding. According to previous research, the fact is that there may also be a lot of other biases that may exist in the research funding, including racial or nationality bias, ethnicity bias, risk aversion bias, etc. [\cite{wojick2015government}] All the diversity should be as important as each other [\cite{LSE2018}], and bias in any aspect would lead to direct impact on the science innovation and career life of individuals. As such, further studies should be conducted on other elements of discrimination to heighten awareness. 

\pagebreak