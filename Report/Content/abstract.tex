\begin{abstract}

The advancement in Science, Technology, Engineering, and Mathematics (STEM) fields has profoundly transformed the global landscape. In response, the UK government has introduced numerous strategies to amplify funding in STEM, aiming to foster innovation and research. However, the equitable distribution of these benefits for different groups of people remains under scrutiny, which could directly affect the incentive of researchers and impact scientific innovation. My research looks into the gender disparities in STEM funding, employing a machine-learning methodology previously unexplored in this domain, thereby addressing this critical gap. I utilised a raw dataset comprising 107,760 projects for this research, where the projects were classified by topic analysis and ChatGPT. The findings were then compared and analysed among various institutions and categories. Though the result proved the existence of a gender disparity in STEM research funding, it displayed a decreasing trend in general. Among the various STEM fields, Agriculture, Forestry \& Food Science exhibited the most pronounced gender bias. Interestingly, Electrical, Electronic \& Computer Engineering stood out as an exception, with a higher funded ratio for females than males. The general pattern of gender disparity could be due to the previously accumulated advantage of males. To address this, a comprehensive approach to enhance the academic structure is necessary, encompassing aspects ranging from promoting diversity in university employment to revising the criteria for funding applications.

\end{abstract}
