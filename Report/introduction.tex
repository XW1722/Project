\section{Introduction}

\noindent Objectivity is fundamental in academic practice to optimally assess one's ability and research findings [\cite{handley2015quality}]. Nevertheless, there is still evidence of gender disparity in many aspects worldwide, where the difference is more prominent among academics working in Science, Technology, Engineering and Mathematics (STEM) [\cite{garcia2019men}]. It has been established that women represent 53\% of bachelor's and master's degree graduates globally, but this proportion decreases to 43\% at the PhD level. Despite this, women hold only 28\% of research positions [\cite{garcia2019men}]. This difference is more noticeable in senior-level positions. The She Figures 2021 data, which studies the gender situation in European countries, has shown that only 26.2\% of the grade-A highest academic positions were held by women [\cite{EUequality}]; female researchers in Ireland accounted for 33\% of full professors according to the data published by Dublin City University (DCU) in 2019 [\cite{hosseini2021gender}]; and this number drops to 25\% full professors in United Kingdom (UK) by the data published by Higher Education Statistics Agency (HESA) in 2018, and 32\% of university governing board members [\cite{thelwall2020gender}].

A less-prominent position for females in the academic field could lead to an unbalanced study environment, exacerbating the shortage of skilled workers in STEM and also adversely affecting the situation of female workers [\cite{verdugo2022gender}]. The UK government Science and Technology Committee reported in 2014, on women's scientific careers in the UK, that \begin{quote} \it ``...the problems and solutions have long been identified, yet not enough is being done to improve the situation''\end{quote} \cite{white2022subject}. This unbalanced environment could result in fewer measures to tackle gender-related problems \cite{garcia2019men}, as previous literature has shown that studies by male researchers are generally less likely to judge gender discrimination than those by females \cite{handley2015quality}. Results within such a biased academic environment can be irrelevant or even harmful to women \cite{cislak2018bias}.
 [\cite{white2022subject}]. This biased environment could result in fewer measures to tackle gender-related problems [\cite{garcia2019men}], as the previous literature has shown that studies by male researchers are generally less likely to judge gender discrimination than those by females [\cite{handley2015quality}]. Results within such a biased academic environment can be irrelevant, or even harmful, to women [\cite{cislak2018bias}]. 

STEM studies, which are fundamental for the resolution of pressing global issues such as climate change and human health, are known to have a considerable gender gap around the world [\cite{garcia2019bridging, Flavia}]. Although there is a growing body of research addressing the diversity gap across multiple facets of the STEM domain (employment, salary, etc.) to ensure the payoffs from STEM are equitable, the impartiality of these studies remains subject to scrutiny due to the current gender difference in researchers. Recent labour market data in 2022/23 indicates that women comprised only about 26\% of the STEM workforce in the UK, suggesting that equal representation may not be achieved until 2070 at the current rate of growth [\cite{womenSTEM}]. This imbalance could exacerbate a cumulative bias known as the "Mathew Effect"[\cite{jebsen2020review}]. For instance, since women are less likely than men to secure positions in higher-ranking institutions, they may face additional challenges when applying for funding [\cite{cruz2022gender}]. Consequently, more research is necessary to address the gender gap in STEM in a comprehensive way.

Currently, there has been an increasing number of studies that have focused on the gender gap in STEM, while the gap still exists from the aspect of STEM funding, and the methodology differs. The review by \cite{verdugo2022gender} found that the number of related literature reached a maximum in 2017 and 2018 [\cite{verdugo2022gender}]. The researchers tried to find the possible reason for the gender differences. The study by \cite{kang2019gender} in Finland found the existence of clear gender differences among children's interests regarding science subjects, while teachers are an important agent that may influence their future path. Another common point is the challenges related to pregnancy and caring responsibilities, which could significantly disadvantage female researchers when returning after a career break [\cite{craig2011non}].
However, there are differences among the methods used in these studies. Among existing studies, quantitative methods are still the most popular analysis method [\cite{verdugo2022gender}]. The study by \cite{delaney2019understanding} applied regression analysis to find differences in the choices and acceptance of STEM courses by gender in Ireland; \cite{kube2024addressing} applied group concept mapping in recent research to see the impact of gender bias in STEM education in Germany, and then aggregated the data through multidimensional scaling and hierarchical cluster analysis; Spatial visualisation is also a popular methodology among present meta-data research in this field [\cite{doi:10.1080/0020739X.2019.1640398}]. However, to our knowledge, machine learning methods have not yet been applied in this field. Furthermore, a majority of current studies focus on the aspect of STEM education and employment to find the gender gap, while comparatively less attention is paid to the funding to STEM, which is indeed an essential factor influencing the incentive of female researchers to produce research.

As one of the top global corporate Research \& Development (R\&D) investors[\cite{GII2022}], the UK government announced a £39.8 billion R\&D budget for 2022-2025 to stimulate the power in science development [\cite{funding2025}]. However, data from the Engineering and Physical Sciences Research Council (EPSRC) in 2016-2017 showed that £944 million of funding was awarded to male applicants while only £69 million went to female applicants [\cite{jebsen2020review}]. An unequal opportunity for research funding would reduce the incentive and access to institutional resources [\cite{cruz2022gender}].  Therefore, in this study, my aim is to determine whether there is a gender bias in STEM from the aspect of research funding, specifically in the UK. For this purpose, I analyze the general distribution of gender ratios and subsequently compare them across different categories and institutions. My primary focus will be to answer the following questions:
\begin{itemize}
    \item From 2015 to 2022, how has the funding trend for each gender evolved? Is there a noticeable disparity?
    \item If there is a gender disparity, what could be the underlying cause?
    \item Is there any implication of the bias?
\end{itemize}
\bigbreak
\noindent To address the first question, this report will assess the overall pattern of funding amount and ratio over time for each gender, and hypothesis testing will be employed for evidence. The subsequent questions will be explored by comparing the trend among different STEM categories, considering recent policies and previous literature.
