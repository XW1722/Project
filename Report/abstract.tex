\begin{abstract}

Advancements in the fields of science, technology, engineering, and mathematics (STEM) have profoundly transformed the global landscape. In response, the UK government has introduced numerous strategies to increase funding in STEM, with the aim of fostering innovation and research. However, the equitable distribution of these benefits for different groups of people remains under scrutiny, which could directly affect the incentive of researchers and impact scientific innovation. My research investigates gender disparities in the amount of funding and proportion of each gender getting funded within STEM disciplines, using a machine learning approach, Natural Language Processing (NLP), which has not been previously applied to address this issue, thus exploring this critical gap. I used a raw data set comprising 107,760 projects for this investigation, where the projects were classified by topic analysis and ChatGPT. The findings were then compared and analysed among various STEM categories in terms of the average amount of funding and the funding ratio of each gender. Although the results confirmed the presence of gender disparity in both the average amount of STEM research funding and the funding ratio for each gender, they also indicated a general decline in bias over time. Among the STEM categories, chemistry stands as an exception, in which the gender gap has increased in recent years. The overall pattern of gender disparity may stem from the historically accumulated advantages of males, including biases in recruitment and promotion processes, differences in life goals, and the absence of specific equality policies, particularly in research funding. To tackle this issue, a comprehensive approach is needed to strengthen the academic structure, including efforts to promote diversity in university employment and revising the criteria for funding applications.

\end{abstract}
